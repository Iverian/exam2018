\documentclass[__main__.tex]{subfiles}

\begin{document}

\section{Методы решений линейных разностных уравнений (2-го порядка) с постоянными коэффициентами и специальной правой частью}

Рассмотрим однородное приведенное линейное разностное уравнение с постоянными коэффициентами второго порядка:

$$
    y_{n+2} + a y_{n+1} + b y_{n} = 0, 
$$

Будем искать частные решения этого уравнения в виде $y_{n}=\lambda^{n}$, где $\lambda\neq 0$ --- неизвестное число. Подставив в исходное уравнение, получим \textit{характеристическое уравнение}:

$$
    \lambda^{2} + a \lambda + b = 0,
$$

Рассмотрим различные случаи решения характеристического уравнения:

- Один вещественный корень кратности два ($\lambda$): получим два частных решения $\lambda^{n}$ и $n\lambda^{n}$, общее решение примет вид:

$$
    y_{n} = \lambda_{n}\left(C_{1} + C_{2}n\right),
$$

- Два вещественных корня ($\lambda_{1}$ и $\lambda_{2}$):

$$
    y_{n} = C_{1}\lambda^{n}_{1} + C_{2}\lambda^{n}_{2},
$$

- Два сопряженных комплексных корня ($\rho\left(\cos\omega \pm i\sin\omega\right)$):

$$
    y_{n} = \rho^{n}\left(C_{1}\sin\left(\omega n\right) + C_{2}\cos\left(\omega n\right)\right),
$$


Теперь рассмотрим соответствующее неоднородное уравнение со специальной правой частью:

$$
    y_{n+2} + a y_{n+1} + b y_{n}
    =
    \left|\lambda\right|^{n}\left(
          p_{n}\cos\left( \omega n \right)
        + q_{n}\sin\left( \omega n \right)
    \right),
$$
где $p_{n}$ и $q_{n}$ --- заданные вещественные многочлены не выше второй степени, $\left|\lambda\right|$, $\omega=\mathrm{const}\in\mathbb{R}$. Для уравнения такого вида имеет место быть следующий метод подбора частного решения:

Если $\xi=\left|\lambda\right|\left( \cos\omega+i\sin\omega \right)$ --- $s$-кратный корень \textit{характеристического уравнения}, то частное решения $\tilde{y}_{n}$ имеет вид:
$$
    \tilde{y}_{n} = n^{s}\cdot\left|\lambda\right|^{s}\cdot
    \left(
          \tilde{p}_{n}\cos\left( \omega n \right)
        + \tilde{q}_{n}\sin\left( \omega n \right)
    \right),
$$
где $\tilde{p}_{n}$ и $\tilde{q}_{n}$ -- многочлены не выше второй степени с неопределенными коэффициентами. Коэффициенты данных многочленов определяются подстановкой частного решения в исходное неоднородное уравнение.

\end{document}
