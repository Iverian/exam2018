\include{preamble}

\begin{document}
\section{Понятия табулирования, базы аппроксимирования сепарабельного банахова пространства, согласованной с его табулированием, интерпретирования табулирования и индуцированного ими аппроксимирования этого пространства. Понятия сходимости, аналитической корректности, устойчивости и корректности такого аппроксимирования.}

Пусть $Y_0 = \left(Y_0; ||\cdot||\right)$ - сепарабельное банахово (гильбертово) пространство и $\hat{F}$ - линейный оператор в $Y_0$. Кроме того, задана последовательность табличных пространств $\left(^>R^{n_k}_{*}:n_k\in N\right)$, где $\lim\limits_{k\to \infty}n_k = +\infty$.
\begin{itemize}
	\item Пусть для каждого $n_k\in N$ задан эпиморфизм $\hat{\pi}_k\in Hom_c\left(Y_0,\;^>R^{n_k}_{*}\right)$ так что $||\hat{\pi}_k|| = 1$. Тогда эпиморфизм $\hat{\pi}_k$ называют $\textbf{табуляцией}$ пространства $Y_0$;
	\item Последовательность табуляций $\hat{\pi}_{(\cdot)} = \left(\hat{\pi}_k\right)_N$ называют $\textbf{табулированием}$ пространства $Y_0$, если для $\forall y_0\in Y_0$ следует, что $\lim\limits_{k\to\infty}||\hat{\pi}_k(y_0)|| = ||y_0||$;
	\item Поскольку $Y_0$ сепарабельное пространство, в нем можно выбрать базу аппроксимирования $H_{(\cdot)} = \left(H_k\right)_N$, где $H_k \subset Y_0$ - конечная система линейно независимых элементов $Y_0$ для $\forall k\in N$. Последовательность $H_{(\cdot)}$ называют \textbf{базой аппроксимирования} пространства $Y_0$, если для $\forall y_0\in Y_0$ существует такая последовательность $y_{(\cdot)} = \left(y_k\in [H_k] = Y_k\right)_N$, что $\lim\limits_{k\to\infty} y_k = y_0$ ($[H_k] = Y_k$ - линейная оболочка системы $H_k$ для $\forall k\in N$);
	\item Пусть $H_{(\cdot)}$ - база аппроксимирования $Y_0$ и $|H_k| = |n_k| = dim\left(\;^>R^{n_k}\right)$ для $\forall k\in N$, где $\hat{\pi}_{(\cdot)} = \left(\hat{\pi}_k\in Hom_c\left(Y_0,\;^>R^{n_k}_{*}\right)\right)_N$ - табулирование пространства $Y_0$. Тогда говорят, что база $H_{(\cdot)} \;\;\textbf{согласована с табулированием}\;\; \hat{\pi}_{(\cdot)}$;
	\item Пусть база $H_{(\cdot)}$ - согласована с табулированием $\hat{\pi}_{(\cdot)}$ и для каждого $k\in N$ определена \textbf{интерпретация}
	\begin{gather*}
		\hat{\varphi}_k\in Hom\left(\;^>R^{n_k}_{*}, [H_k] = Y_k\right)
	\end{gather*} 
	Последовательность $\hat{\varphi}_{(\cdot)} = \left(\hat{\varphi}_k\right)_N$ называют \textbf{интерпритированием} если $\lim\limits_{k\to\infty}||\hat{\varphi}_k|| = 1$ ($\hat{\varphi}_{(\cdot)}$ - интерпритирование для табулирования $\hat{\pi}_{(\cdot)}$). В этом случае для каждого $k\in N$ определена $\textbf{аппроксимация}\;\; \hat{p}_k = \hat{\varphi}_k \circ \hat{\pi}_k \in Hom_c\left(Y_0, [H_k] = Y_k\right)$ элементов пространства $Y_0$, т.е $\hat{p}_k(y_0)$ - аппроксимация $y_0 \in Y$ для $\forall y_0\in Y$. Последовательность $\hat{p}_{(\cdot)} = \left(\hat{p}_k\right)_N$ называют $\textbf{аппроксимированием}$ пространства $Y_0$;
	\item Если для $\forall y_0\in Y_0 \;\;\exists \lim\limits_{k\to\infty}\hat{p}_k(y_0)$, то аппроксимирование $\hat{p}_{(\cdot)}$ называют \textbf{сходящимся};
	\item Если $\hat{p}_{(\cdot)}$ - сходящиеся аппроксимирование и $\lim\limits_{k\to\infty}\hat{p}_k(y_0) = y_0$, то аппроксимирование $\hat{p}_{(\cdot)}$ называют \textbf{корректным};
	\begin{gather*}
	||\hat{p}_k|| = ||\hat{\varphi}_k \circ \hat{\pi}_k|| \leq ||\hat{\varphi}_k|| \cdot ||\hat{\pi}_k|| \leq ||\hat{\varphi}_k|| \rightarrow 1	
	\end{gather*}
	при $k\to\infty$, т.е $\hat{p}_{(\cdot)}$ - \textbf{устойчиво} и \textbf{аналитически корректно}.
\end{itemize}


\end{document}