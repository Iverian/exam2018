\documentclass[__main__.tex]{subfiles}

\begin{document}

\section{Теорема о норме обратной матрицы к матрице, имеющей диагональное преобладание}

27+ страницы Лекций Димы.\\

Для матрицы $F \in \mathbb{L}(\mathbb{R};k+1)$ вида

\begin{equation}
\label{eq_11_2}
A = \begin{pmatrix}
b_0 & c_0 & 0 & \hdots & \hdots & \hdots & 0 \\
a_1 & b_1 & c_1 & 0 & \hdots & \hdots & 0 \\
0 & a_2 & b_2 & c_2 & 0 & \hdots & 0 \\
0 & \hdots & \hdots & \hdots & \hdots & \hdots & 0 \\
0 & \hdots & \hdots & 0 & a_{k-1} & b_{k-1} & c_{k - 1}\\
0 & \hdots & \hdots & 0 & 0 & a_k & b_k\\ 
\end{pmatrix}
\end{equation}

Если
\begin{equation}
|b_i| - |a_i| - |c_i|\ge J > 0, \text{для} i = \overline{1, k} \;\;\; (a_0 = c_k = 0)
\end{equation}
, то $F \in \mathbb{GL}(\mathbb{R};k+1)$ и $||F^{-1}||\le \frac{1}{J}$\\



\begin{theorem}
Пусть матрица $A \in \mathbb{L}(\mathbb{R};n)$ имеет диагональное перобладание 
$$diag(A)=min\{|a^j_j|-\sum_{j\neq i} |a_j^i|\}>0$$
Тогда $A \in \mathbb{GL}(\mathbb{R};n)$ и $||A^{-1}||\le \frac{1}{diag(A)}$
\begin{proof}
Рассмотрим совместную СЛАУ:
$A^>x=^<b$, где $^>x\in ^>\mathbb{R}^n$- решение СЛАУ, где $||^>x||=max_i(|x^i|)$

\begin{equation}
\begin{cases}
a_1^1x^1=b^1-\sum_{k\neq 1} a_k^1 x^k\\
a_2^2x^2=b^2-\sum_{k\neq 2} a_k^2 x^k\\
.\\
.\\
.\\
a_n^n x^n=b^n-\sum_{k\neq n} a_k^n x^k\\
\end{cases}
\end{equation}

\begin{equation}
\begin{cases}
|a_1^1||x^1|\le |b^1|+\sum_{k\neq 1} |a_k^1| |x^k|\\
|a_2^2||x^2|\le |b^2|+\sum_{k\neq 2} |a_k^2| |x^k|\\
.\\
.\\
.\\
|a_n^n| |x^n|\le |b^n|+\sum_{k\neq n} |a_k^n| |x^k|\\
\end{cases}
\end{equation}
В одной из строк $|x^j|=||^>x||$, а значит:
$$(|a^j_j|-\sum_{k\neq j} |a_k^j|)||^>x||\le|b^j|\le||^>b||$$
С другой стороны $diag(A) ||^>x||\le(|a^j_j|-\sum_{k\neq j} |a_k^j|)||^>x||$, а значит, в силу неотицательности диагонального преобладания,

$$||^>x||\le \frac{1}{diag(A)},$$

в свою очередь стоит отметить, что ${}^>x=A^{-1} {}^>b$, из чего следует, что 

$$||A^{-1}||\le\frac{1}{diag(A)}$$
чтд
\end{proof}
\end{theorem}
\end{document}
