\documentclass[__main__.tex]{subfiles}

\begin{document}

\section{Метод матричной прогонки для решения СЛАУ с трёхдиагональной матрицей, имеющей диагональное преобладание}

Рассмотрим СЛАУ:

\begin{equation}
\label{eq_11_1}
A \cdot {}^{>}x = {}^{>}f
\end{equation}

где $A = (a^i_j)^n_n \in L(\mathbb{R}, n)$ невырожденная квадратная матрица,
где $f = [f_1, f_2, \text{...}, f_n> \in {}^{>}\mathbb{R}^n$ - вектор-столбец правой части СЛАУ и ${}^{>}x = [x_1, x_2, \text{...}, x_n>$ - вектор-столбец неизвестных.\\

Пусть в СЛАУ \ref{eq_11_1} матрица $A = (a^i_j)^n_n \in L(\mathbb{R}, n)$ имеет трехдиагональный вид, т.е.:

\begin{equation}
\label{eq_11_2}
A = \begin{pmatrix}
b_1 & c_1 & 0 & \hdots & \hdots & \hdots & 0 \\
a_2 & b_2 & c_2 & 0 & \hdots & \hdots & 0 \\
0 & a_3 & b_3 & c_3 & 0 & \hdots & 0 \\
0 & \hdots & \hdots & \hdots & \hdots & \hdots & 0 \\
0 & \hdots & \hdots & 0 & a_{n-1} & b_{n-1} & c_{n - 1}\\
0 & \hdots & \hdots & 0 & 0 & a_n & b_n\\ 
\end{pmatrix}
\end{equation}

где 
\begin{equation}
\label{eq_11_3}
|b_i| - |a_i| - |c_i| > 0, \text{для} i = \overline{1, n} \;\;\; (a_1 = c_n = 0)
\end{equation}

Согласно условию СЛАУ \ref{eq_11_3}, матрица $A$ - невырождена и, следовательно, СЛАУ (1) имеет единственное решение.

\subsection{Прямой ход метода прогонки}

Благодаяры трехдиагональности матрицы СЛАУ \ref{eq_11_2}, прямой ход метода прогонки является методом последовательного исключения неизвсетных. Из  первого уравнения СЛАУ выражается первое неизвестное через второе, и это выражение подставляется в первое неизвестное второго уравнения. Таким образом, из прямого хода получают равенства

\begin{equation}
\label{eq_11_4}
\begin{cases}
x_n = L_k x_{k+1} + M_k, k = \overline{1, n-1}\\
x_n = M_n
\end{cases}
\end{equation}

\begin{equation}
\label{eq_11_5}
\begin{cases}
L_1 = -\frac{c}{b}, \; M_1 = \frac{f_1}{b_1}\\
L_k = -\frac{c_k}{L_{k-1} a_k + b_k}, M_k = \frac{f_k - M_{k-1} a_k}{L_{k-1} a_k + b_k}, k = \overline{2, n - 1}\\
M_n = \frac{f_n - M_{n-1} a_n}{L_{n-1}a_n + b_n} = x_n\\
\end{cases}
\end{equation}

\subsection{Обратный ход метода прогонки}

Согласно прямому ходу метода прогонки \ref{eq_11_5}, в равенствах \ref{eq_11_4} для $k = \overline{1, n - 1}$ вычислены все коэффициенты $L_k$ и $M_k$ и последняя компонента $x_n = M_n$ вектора неизвестных СЛАУ \ref{eq_11_1} с матрицей вида \ref{eq_11_2}. Эти найденные числа позволяют последовательно найти все остальные компоненты вектора неизвестных СЛАУ \ref{eq_11_4} с матрицей вида \ref{eq_11_2} из равенств:

\begin{equation}
\begin{cases}
x_{n-1} = L_{n-1}x_n + M_{n-1}\\
x_{n-2} = L_{n-2}x_{n-1} + M_{n-2}\\
\hdots\\
x_1 = L_1 x_2 + M_1;
\end{cases}
\end{equation}

Тем самым получаем решение СЛАУ.

\end{document}
