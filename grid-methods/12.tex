\documentclass[__main__.tex]{subfiles}

\begin{document}

\section{Корректная конечно-разностная схема численного решения одномерной краевой задачи (граничные условия 1-го рода) для одномерного стационарного уравнения теплопроводности с постоянным коэффициентом теплопроводности и положительной теплоотдачей}

Рассмотрим краевую задачу:
\begin{gather}
	\begin{cases}
		-u''(\tau)+q(\tau)u(\tau)=f(\tau), & \tau\in[a;b];\\
		u(a) = u_a \in \mathbb{R} & u(b) = u(b) \in \mathbb{R}, 
	\end{cases}
	\label{12-1}
\end{gather}
где $q\ge0$ на $[a;b]$\\
Используем на $[a,b]$ схему равномерных сеток $A_{(.)}=(A_k=\langle a=\tau_0,\tau_1,...,\tau_k\rangle )_{\mathbb{N}}$, где $h=stp(A_k)$, согласно разностным производным из уравнения $\ref{12-1}$ получаем конечно-разностную схему:
\begin{gather}
	\begin{cases}
		u_0 = u_a;\\
		-\frac{u_{i+1}-2u_i+u_{i+1}}{h^2} + q_iu_i = f_i, & i=\overline{1,k-1};\\
		u_k=u_b,
	\end{cases}
	\label{12-2}
\end{gather}	
где $\;^>u_{(k)} = [u_0,u_1,...,u_k\rangle, \hat{A}_k(q) = [q_0,q_1,...,q_k\rangle,\\
\hat{A}_k(f) = [f_1, f_2,...,f_{k-1}\rangle \in \;^>\underline{\mathbb{R}}^{|A_k|}(A_k)$\\

Введём обозначения:
\begin{gather}
\begin{cases}
\;^>\nu = [u_a,f_1,f_2,...,f_{k-1},u_b\rangle \in \;^>\underline{\mathbb{R}}^{|A_k|}(A_k);\\
F_{(k)}=
\begin{pmatrix}
b_0 & c_0 & 0 & ... & ... & ... & 0 \\
a_1 & b_1 & c_1 & 0 & ... & ... & 0\\
0 & a_2 & b_2 & c_2 & 0 & ... & 0\\
. & . & & . & & & .\\
. & & . & & . &  & .\\
. & & & . & & . & .\\
0 & ... & ... & ... & ... & ... & 0\\
0 & ... & ... & 0 & a_{k-1} & b_{k-1} & c_{k-1}\\
0 & ... & ... & ... & 0 & a_k & b_k
\end{pmatrix}\\
a_0=0, \; b_0=1, \; c_0=0;\\
a_i=-\frac{1}{h^2}, \; b_i=\frac{2}{h^2}+q_i, \; c_i=-\frac{1}{h^2},\; i=\overline{1,k-1};\\
a_k=0, \; b_k=1,\;c_k=0;
\end{cases}
\label{12-3}
\end{gather}
Используя обозначения $\ref{12-3}$, схему $\ref{12-2}$ представим в виде:\\
\begin{equation}
	F_{(k)}\;^>u_{(k)} = \;^>\nu_{(k)}
	\label{12-4}
\end{equation}
Из $ref{12-3}$ для матрицы $F_{(k)}$ в схеме $\ref{12-4}$ получаем соотношения:
\begin{equation}
	\begin{cases}
		a_i<0, b_i>0 \;(q\ge 0 \; \text{на} \; [a;b]), \; c_i < 0 \; \text{для} \; i=\overline{1,k-1};\\
		a_j+b_j+c_j \ge 0, \; j=\overline{0,k}
	\end{cases}
	\label{12-5}
\end{equation}
Схема $\ref{12-4}$  аппроксимирует уравнение $\ref{12-1}$ со 2-ым порядком аппроксимирования. К тому же СЛАУ $F_{(k)}\;^>u_{(k)} = \;^>\nu_{(k)}$ имеет единственное решение.\\
Также, в силу леммы(принцип максимума для конечно-разностной схемы $\ref{12-4}$), следует, что схема $\ref{12-4}$ устойчива. Поэтому, согласно теореме Лакса, схема $\ref*{12-4}$ корректна, то есть аналитически корректна.
\end{document}
