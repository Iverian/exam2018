\documentclass[__main__.tex]{subfiles}

\begin{document}

\section{Теорема о невырожденности трёхдиагональной матрицы, удовлетворяющей условию квази-диагонального преобладания}


\textbf{Определение квази-диагональной матрицы}
	Если трехдиагональная матрица $G_k = \left(
	\begin{matrix}
	b_1 & c_1 & 0 & 0 & ... & 0 \\
	a_2 & b_2 & c_2 & 0 &... & 0 \\
	0 & a_3 & b_3 & c_3 & ... & 0 \\
	... & ... & ... & ... & ... & ...\\
	0 & ... & 0 & a_{n-1} & b_{n-1} & c_{n-1} \\
	0 & ... & 0 & 0 & a_{n} & b_{n}
	\end{matrix}
	\right)$ удовлетворяет условиям: $ |B_j|  \geq  |C_j|, |B_j|  \geq |A_j| + |C_j| > |A_j|, j = \overline{0,k}$, то говорят, что матрица $G_k$ имеет квазидиагональное преобладание.

	\paragraph{Теорема о невырожденности трёхдиагональной матрицы, удовлетворяющей условию квази-диагонального преобладания}
	
		Матрица $G_k$, определенная выше, невырождена.

\paragraph{Доказательство}
	Рассмотрим СЛАУ:
	\begin{equation}
	\label{13.1}
	G_k \cdot \;^{>}U_{k} = \;^{>}o_{(n+1)}
	\end{equation}
	
	 где $\;^{>}o_{(n+1)}$ - нулевой вектор.
	 
	 Для решения СЛАУ (1) используем метод прогонки:
	 
	 \begin{equation}\label{13.2}
	 \begin{cases}
	 U_j = L_j U_{j+1} + M_j , j = \overline{0,k-1} \\
	 U_k = M_k,
	 \end{cases}
\end{equation}

где 
\begin{equation} \label{13.3}
\begin{cases}
L_j = \frac{C_{0}}{B_0} & M_0 = 0 \\
L_k = \frac{C_{j}}{L_{j-1}A_j + B_j},\\
M_j = \frac{-M_{j-1}A_j}{L_{j-1}A_j + B_j},\\ 
j = \overline{0,k-1} & M_k = \frac{M_{k-1}A_k}{L_{k-1}A_k + B_k},
\end{cases}
\end{equation}

Покажем, что $|L_i| \leq 1$ для $i = \overline{0,k-1}$. Действительно, поскольку $|B_0|\geq|C_0|$, то $|L_0| = $$\frac{C_0}{B_0}$$ \leq 1$. Пусть $|L_{j-1}| \leq 1$. Покажем, что  $|L_{j}| \leq 1$, где $L_j = $$\frac{C_{j}}{L_{j-1}A_j + B_j}$ и $|L_{j-1}A_j + B_j|  \geq |B_j| - |L_{j-1}A_j| \geq$ (поскольку $|L_{j-1}| \leq 1$) $ |B_j|-|A_j| \geq |C_j|,$ в силу квазидиагонального преобладания матрицы $ G_k$.

Следовательно, $L_j = \frac{C_{j}}{L_{j-1}A_j + B_j} \geq 1$ и $|L_{j-1}A_j + B_j| \geq |C_j| > 0$, так как $|A_j + C_j|  > |A_j|$. Поэтому из (\ref{13.2}) получаем единственное решение СЛАУ (\ref{13.1}), т.е. матрица $G_k$ - невырождена.

\paragraph{Следствие}

СЛАУ $F_k \cdot \;^{>}u_k = \;^{>}\nu_k$ (подробнее смотреть "Корректная конечно-разностная схема для одномерного уравнения теплопроводности") имеет единственное решение.

\end{document}
