\documentclass[__main__.tex]{subfiles}

\begin{document}

\section{Построение конечно-разностной схемы, аппроксимирующей общую одномерную краевую задачу (граничные условия 1-го рода) для стационарного уравнения теплопроводности с переменным коэффициентом теплопроводности, неотрицательной теплоотдачей и "конвекцией". Условие квази-диагонального преобладания для трёхдиагональных матриц СЛАУ блоков этой схемы}

Рассмотрим стационарное уравнение теплопроводности:

\begin{gather}
	\begin{cases}
		-\frac{d}{d\tau}(k(\tau)\frac{du}{d\tau})+v(\tau)\frac{du}{d\tau}+q(\tau)u(\tau)=f(\tau), \tau \in (a;b); \\
		u(0)=u_{a}, u(b)=u_{b}.
		\end{cases}
\end{gather}

где $k(\tau)>k_{0}>0 \  q(\tau) \geq 0.$

Конечно-разностная схема для данной задачи:
\begin{gather}
	\begin{cases}
		^>u_{\cdot}=[u_{\cdot}, u_{1},...,u_{n}>\in ^>\mathbf{R}^{|A|}(A)\\
		u_{0}=u_{a},\\
		-\frac{1}{h}(k_{i+\frac{1}{2}}\frac{u_{i+1}-u_{i}}{h}-k_{i-\frac{1}{2}}\frac{u_{i}-u_{i-1}}{h})+v_{i}\frac{u_{i+1}-u_{i-1}}{2h}+u_{i}q_{i}=f_{i}, \ i=1,...,k-1,\\
		u_{k}=u_{b}.
	\end{cases}
\end{gather}
 где $\tau_{i-\frac{1}{2}}=\frac{\tau_{i-1}+\tau_{i-1}}{2}, k_{i-\frac{1}{2}}=k(\tau_{i-\frac{1}{2}}),\tau_{i+\frac{1}{2}}=\frac{\tau_{i+1}+\tau_{i}}{2}, k_{i+\frac{1}{2}}=k(\tau_{i+\frac{1}{2}})$ для i=1..k-1. 
 
 Для данной схемы введем обозначения (заменим систему на СЛАУ $F_{k} ^>u_{k}=^>\nu_{k}$):
 \\
 \begin{gather}
 	\begin{cases}
 		a_{0}=0, b_{0}=1,c_{0}=0;\\
 		a_{i}=-\frac{1}{h^2}k_{i-\frac{1}{2}-\frac{v_{i}}{2h}}\\
 		b_{i}=\frac{1}{h^2}k_{i+\frac{1}{2}}+\frac{1}{h^2}k_{i-\frac{1}{2}}+q_{i};\\
 		c_{i}=-\frac{1}{h^2}k_{i+\frac{1}{2}}+\frac{v_{i}}{2h}, i=1..k-1;\\
 		a_{k}=0, b_{k}=1.
 		\end{cases}
 	\end{gather}
 Таким образом представим схему в виде:\\
 \begin{gather}
 	\begin{cases}
 	F_{k}\cdot ^>u_{(k)}=^>\nu_{k}(k),\\
 	k\to +\infty , h\to0.
 	\end{cases}
 \end{gather}
 
 где 
$ F_{(k)}$=
 $\left(
 \begin{matrix}
 	b_{0} & c_{0} & 0 & 0 & ... & 0\\
 	a_{1}&b_{1}&c_{1} & 0& ...&0\\
 	0&a_{2}&b_{2}&c_{2}&0&...0\\
 	0&...&0&0&0&0\\
 	0&...&0&a_{k-1}&b_{k-1}&c_{k-1}\\
 	0&...&0&0&a_{k}&b_{k}	
 \end{matrix}
 \right)$
 
 тогда $a_{i}+b_{i}+c_{i}=q_{i}\geq0, b_{i}>0,c_{i}<0, k>k_{0}>0$. Следовательно, при $h\to+\infty$ матрица $F_{k}$ является трехдиагональной и удовлетворяет условию квази-диагонального преобладания:\\
 $|b_{i}|\geq|a_{i}|+|c_{i}|,|b_{i}|>|a_{i}|$ для i=0..k.
 


\end{document}
