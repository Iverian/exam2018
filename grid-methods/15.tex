\documentclass[__main__.tex]{subfiles}

\begin{document}

\section{Лемма и теорема о принципе максимума для конечно-разностной схемы, аппроксимирующей краевую задачу (граничные условия 1-го рода) для одномерного стационарного уравнения теплопроводности с постоянным коэффициентом теплопроводности, неотрицательной теплоотдачей (без "конвекции")}

Пусть решение ${}^{>}u_{(k)} = [u_0, u_1, \hdots, h_k>$ СЛАУ $F_(k) * {}^{>}u_(k) = \nu_{(k)}$ - таково, что ${}^{>}\nu_{(k)} \le {}^{>}O_{k+1}$, т.е. $\nu_j \le 0$ для $\forall j = \overline{0, k}$. Тогда ${}^{>}u_{(k)} \le {}^{>}O_{k+1}$.
\textbf{Доказательство (от противного)}

Предположим противное, т.е. $M = \max\{u_i: i = \overline{1, k - 1}\} = u_j > 0$, где $j$ - максимальный индекс среди индексов от 1 до $k-1$, для которого $u_j = M$. Тогда $u_{j-1} \le u_j$ и $u_{j+1} \le u_j$.

Поскольку $a_i \ge 0$ и $c_i \ge 0$ (см. вопрос 12), то

\begin{equation}
\begin{cases}
a_{j - 1} u_{j - 1} \le a_j u_j \\
c_{j + 1} u_{j + 1} > c_j u_j \\
a_i + b_i + c_i \ge 0 \text{для} i = \overline{1, k - 1}\\
\end{cases}
\end{equation}

Тогда из схемы ($F_{(k)} \cdot {}^{>}u_(k) = \nu_{(k)}$) для $i = j$ получаем ($\nu_j = f_j \le 0$ из условия Леммы):

$$0 \ge f_j = a_{j - 1} u_{j - 1} + b_j u_j + c_{j + 1} u_{j + 1} >\{\text{см. вопрос 10}\} > a_j u_j + b_j u_j + c_j u_j = (a_j + b_j + c_j)u_j \ge 0$$
, т.е $0 > 0$ - противоречие.

Таким образом, получаем, что ${}^{>}u_{(k)} = {}^{>}O_{k+1}$, где $O_{k+1} = [0, ..., 0> \in {}^{>}\mathbb{R}^{|A_k|}(A_k)$.

\end{document}
