\documentclass[__main__.tex]{subfiles}

\begin{document}

\section{Теоремы сравнения для сеточных функций сравнения, необходимые для доказательства устойчивости конечно-разностной схемы, аппроксимирующей одномерную краевую задачу (граничные условия 1-го рода) для стационарного уравнения теплопроводности с постоянным коэффициентом теплопроводности, неотрицательной теплоотдачей (без "конвекции")}

Есть обращения к билету 12 и 15.

\paragraph{Предисловие}

Рассмотрим краевую задачу:

\begin{equation}
\begin{cases}
-U''\left(\tau\right)+q\left(\tau\right) U \left(\tau\right) = f\left( \tau \right), \ \tau \in [a;b] \\
U\left(a\right) = U_a \in \mathbb{R}, \ U\left(b\right) = U_b \in \mathbb{R}
\end{cases}
\end{equation}

где $q \geq 0$ на $[a;b]$.

\paragraph{Теорема сравнения}

Рассмотрим СЛАУ:

\begin{equation} \label{16.1}
F_{\left(k\right)} \cdot {}^> U_{\left(k\right)} = {}^> \nu_{\left(k\right)}
\end{equation}

и соотношения

\begin{equation} \label{16.2}
\begin{cases}
a_i < 0, b_i > 0, c_i < 0, i = \overline{1,k-1} \\
a_j+b_j+c_j \geq 0, j =\overline{0,k}.
\end{cases}
\end{equation}

Пусть кроме СЛАУ \ref{16.1}, где матрица $F_{\left(k\right)}$ удовлетворяет условиям \ref{16.2}, рассматриваеется ещё одна СЛАУ:

\begin{equation}\label{16.3}
F_{\left(k\right)} \cdot {}^> x_{\left(k\right)} = {}^> y_{\left(k\right)}.
\end{equation}

Для решения ${}^> x_{\left(k\right)} = [x_0, x_1, ..., x_k\rangle \in {}^> \mathbb{R}^{\left|A_k\right|} \left(A_k\right)$ СЛАУ \ref{16.3} и решения СЛАУ \ref{16.1} выписываются условия:

\begin{equation}\label{16.4}
\left| \left(F_{\left(k\right)} \cdot {}^> U_{\left(k\right)}\right)_i \right| = \left|\nu_i\right| \leq \left(F_{\left(k\right)} \cdot {}^> x_{\left(k\right)}\right)_i = y_i, \ i = \overline{0,k}.
\end{equation}

Тогда $\left|U_i\right| \leq x_i$ для $i = \overline{0,k}$, где ${}^> U_{\left(k\right)} = [ U_0, U_1, ..., U_k \rangle$.

\paragraph{Доказательство}

Рассмотрим сеточную функцию ${}^> z_{\left(k\right)} = - {}^> \nu_{\left(k\right)} - {}^> y_{\left(k\right)}$. Тогда, согласно \ref{16.4}: $- {}^>\nu_{\left(k\right)} - {}^> y_{\left(k\right)}\leq O_{\left(k+1\right)}$. 

Следовательно, из принципа макисмума получаем:

\begin{equation} \label{16.5}
- {}^> U_{\left(k\right)} - {}^> x_{\left(k\right)} \leq {}^> O_{\left(k+1\right)} \Leftrightarrow - U_i \leq x_i, \ i = \overline{0,k}.
\end{equation}

Рассмотрим сеточную функцию:

$$
{}^> \mu_{\left(k\right)} = - {}^> \nu_{\left(k\right)} + {}^> y_{\left(k\right)} \leq {}^> O_{\left(k+1\right)}.
$$

Тогда из принципа максимума следует:

\begin{equation}\label{16.6}
- {}^> x_{\left(k\right)} + {}^> U_{\left(k\right)} \leq {}^> O_{\left(k+1\right)} \Leftrightarrow U_i \leq x_i, \ i= \overline{0,k}.
\end{equation}

Таким образом из \ref{16.5} и \ref{16.6} получаем: $\left|U_i\right| \leq x_i, \ i = \overline{0,k}$. $\ \ \#$

\paragraph{Лемма 1}

Рассмотрим частный случай задачи из предисловия:

\begin{equation}\label{16.7}
\begin{cases}
-U''+q\left(\tau\right) U =0, \ \tau \in \left(a;b\right) \\
U\left(a\right) = U_a, \ U\left(b\right) = U_b
\end{cases}
\end{equation}

Тогда схема \ref{16.1} имеет вид:

\begin{equation}\label{16.8}
F_{\left(k\right)} \cdot {}^>U_{\left(k\right)} = \left(
\begin{matrix}
U_a \\ 0 \\ 0 \\ ... \\ 0 \\ U_b
\end{matrix}
\right)
\end{equation}

В этому случае: $\| {}^>U_{\left(k\right)} \| \leq max \{ \left|U_a\right|, \left|U_b\right| \}$, если ${}^>U_{\left(k\right)}$ - решение \ref{16.8}.

\paragraph{Доказательство}

Пусть $M = max \{ \left|U_a\right|, \left|U_b\right| \}$ и ${}^> x_{\left(k\right)} = [y_0, y_1, ..., y_k \rangle = [M,M,...,M \rangle \in {}^> \mathbb{R}^{\left|A_k\right|} \left(A_k\right)$. Тогда 

$$
F_{\left(k\right)} \cdot {}^>x_{\left(k\right)} = \left(
\begin{matrix}
M \\ q_1 M \\ q_2 M \\ ... \\ q_{k-1} M \\ M
\end{matrix}
\right) \geq \left(
\begin{matrix}
U_a \\ 0 \\ 0 \\... \\ 0 \\ U_b
\end{matrix}
\right)
$$

Тогда согласно принципа максимума для решения ${}^> U_{\left(k\right)}$ СЛАУ \ref{16.8} получаем:

$$
\left|U_i\right| \leq M = max \{ \left|U_a\right|, \left|U_b\right| \}, \ i = \overline{0,k}
$$

то есть $\| {}^> U_{\left(k\right)} \| \leq max \{ \left|U_a\right|, \left|U_b\right| \}$. $\ \ \# $]

\paragraph{Лемма 2}

Другой частный случай:

\begin{equation}\label{16.9}
\begin{cases}
-U''+q\left(\tau\right) U = f\left(\tau\right), \ \tau \in \left(a;b\right) \\
U\left(a\right) = 0, \ U\left(b\right) = 0
\end{cases}
\end{equation}

Тогда 

\begin{equation}\label{16.10}
F_{\left(k\right)} \cdot {}^>U_{\left(k\right)} = \left(
\begin{matrix}
0 \\ f_1 \\ ... \\ f_{k-1} \\ 0
\end{matrix}
\right) = {}^>\nu_{\left(k\right)}
\end{equation}

В этом случае: $\| {}^> U_{\left(k\right)} \| \leq \frac{\left(b-a\right)^2}{8} max \{ \left|f_1\right|, \left|f_2\right|, ..., \left|f_{k-1}\right| \}$, если ${}^>U_{\left(k\right)}$ - решение \ref{16.10}.

\paragraph{Доказательство}

Пусть $C = max \{\left|f_1\right|,\left|f_2\right|,...,\left|f_{k-1}\right|\}$.

Рассмотрим функцию: $x\left(\tau\right) = \frac{C}{2} \left(\tau - a\right) \left(\tau - b\right), \ \tau \in [a;b]$.

Очевидно, что $x\geq 0$ на $[a;b]$ и, если ${}^> x_{\left(k\right)} = [ x_0 = 0, x_1, ..., x_{k-1}, x_k = 0 \rangle = \hat{A}_k \left(x\right)$, то, по аналогии с \ref{12-2} из вопроса №12, получаем:

$$
\begin{cases}
x_0 = 0 \\
-\frac{x_{i-1} - 2 x_i + x_{i+1}}{h^2} + q_i x_i = C + q_i x_i \geq \left|f_i\right|, \ i = \overline{1,k-1} \\
x_k = 0
\end{cases}
$$

Согласно предыдущей лемме, для этой СЛАУ и СЛАУ \ref{16.10} получаем: $\| {}^> U_{\left(k\right)} \| \leq max \{ x \left(\tau\right): \tau \in [a;b]\} = max \{\frac{C}{2}\left(\tau - a\right) \left(b- \tau\right): \tau \in [a;b] \} = \big| x_{min} = \frac{a+b}{2} \big| = C \cdot \frac{\left(b-a\right)^2}{8}$. $\ \ \#$
\end{document}