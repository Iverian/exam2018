\documentclass[__main__.tex]{subfiles}

\begin{document}

\section{Матрица Бутчера и общий вид методов Рунге Кутта}

Рассомтрим задачу Коши:
\begin{gather}
	\begin{cases}
		\frac{dy}{d\tau}=f(\tau,y),\; \tau\in(0,t)\\
		y(0)=y_0
	\end{cases}
	\label{K}
\end{gather}
где $f$ - достаточно гладкая функция, т.е. решение \ref{K} существует, единственно и достаточно гладкое.

Для этого введём схему равномерных сеток $A_{(.)}=(A_k)_\mathbb{N}$, где $A_n=<0=\tau_0,\tau_1,\dots,\tau_k=t>$ - равномерная сетка с шагом $h=t/k.$ На отрезке $[\tau_0{i-1},\tau_i]$ уравнение из \ref{K} равносильно ур-ю:
\begin{gather}
	y_i=y_{i-1}+\int\limits_{\tau_{i-1}}^{\tau_i}f(\tau,y(\tau))d\tau,
	\label{rv}
\end{gather}
где $\;^>y=[y_0,y_1,\dots,y_k>=\hat A_k(y).$

Поскольку $\tau_i=\tau_{i-1}+h$, то, используя квадратурную формулу трапеций, получаем из \ref{rv} приближенное равенство:
\begin{gather}
	y_i=y_{i-1}+\frac{h}{2}\cdot\left(f(\tau_{i-1},y_{i-1})+f(\tau_{i},y_{i})\right)+O(h^3),\; h\rightarrow0
	\label{sh}
\end{gather}
Используем метод Эйлера для значения $f(\tau_i,y_i):$
\begin{gather}
	f(\tau_i,y(\tau_i))\approx f(\tau_{i-1}+h,y_{i-1}+hf(\tau_{i-1},y_{i-1}))
	\label{E}
\end{gather}

Из \ref{sh} и \ref{E} для метода Рунге-Кутта 2-го порядка (Эйлера-Коши) получаем рабочую формулу:
\begin{gather}
	y(\tau+h)=y(\tau)+h\left[\frac{1}{2}f(\tau,y(\tau))+\frac{1}{2}f(\tau+h,y(\tau))+hf(\tau,y(\tau))\right]+O(h^3),\; h\rightarrow0
	\label{rk}
\end{gather}

Введём обозначения:
\begin{gather}
	\begin{cases}
	\;^>j=<j_1,j_2]=<1/2,1/2]\\
	y=y(\tau)\\
	\omega_1=\omega_1(\tau,y,h)=f(\tau,y)\\
	\omega_2=\omega_2(\tau,y,h)=f(\tau+h,y+hf(\tau,y))\\
	\;^>\omega=[\omega_1,\omega_2>
	\end{cases}
	\label{o}
\end{gather}
Тогда формулу \ref{rk}, согласно обозначениям \ref{o}, можно получить:
\begin{gather}
	y(\tau)+h=y+\frac{h}{2}(\omega_1+\omega_2)+O(h^3),\; h\rightarrow0
	\label{rf}
\end{gather}
Рабочая формула \ref{rf} индуцирует конечно-разностную схему (схему табличных аналогов \ref{K}), аппроксимирующую \ref{K} со 2-ым порядком аппроксимирования:
\begin{gather*}
	\begin{cases}
	U_0=y_0\\
	U_1=y_0+h\cdot\;^<j\cdot\;^>\omega(\tau_0,y_0,h).
	\end{cases}
\end{gather*}

Для метода Рунге-Кутта порядка $m\in\mathbb{N}$ подбираетя рабочая формула:
\begin{gather*}
	y(\tau+h)=y+h\cdot\;^<j\cdot\;^>\omega(\tau_0,y_0,h)+O(h^{m+1}),\; h\rightarrow0
\end{gather*}
где
\begin{gather*}
	\begin{cases}
		\;^>j=<j_1,j_2,\cdots,j_m]\in\;^>\mathbb{R}^{m+1}:\sum\limits_{i=1}^m j_i=1\\
		y=y(\tau)\\
		\omega_1=\omega_1(\tau,y,h)=f(\tau,y)\\
		\cdots\\
		\omega_m=f(\tau+\alpha_mh,y+h\sum\limits_{j=1}^{m-1}\beta_m^j\omega_j)
	\end{cases}
\end{gather*}
где коэффициенты заданы в матрице Бутчера этого метода Рунге-Кутта порядка $m:$
\begin{gather*}
	B=
	\begin{pmatrix}
	j_1 & \alpha_2 & \alpha_3 & \cdots & \alpha_m \\
	j_2 & \beta_2^1 & 0 & \cdots & 0 \\
	\vdots & \beta_3^1 & \beta_3^2 & 0\cdots & 0 \\
	j_m & \beta_m^1 & \beta_m^2 & \cdots & \beta_m^{m-1}
	\end{pmatrix}
\end{gather*}
\end{document}
