\include{preamble}

\begin{document}
\section{Методы Рунге-Кутта 2-го порядка для численного решения задачи Коши с нормальным ОДУ}

Рассмотрим задачу Коши:
\begin{equation*}
	\begin{cases}
		\frac{dy}{d\tau} = f(\tau,y),\;\;\tau\in [0,t]\\
		y(0) = y_0
	\end{cases}
\end{equation*}
где $f$ - достаточно гладкая функция.

Марица Бутчера для Рунге-Кутты второго порядка выглядит следующим образом:
\begin{gather*}
	B = \begin{pmatrix}
		\gamma_1 = 1 & \alpha_2 = \frac{1}{2}\\
		\gamma_2 = 0 & \beta^1_2 = \frac{1}{2}
	\end{pmatrix}
\end{gather*}
Тогда формула Рунге-Кутты 2-ого порядка будет:
\begin{gather*}
	y(\tau+h) = y(\tau)+\int\limits_{\tau}^{\tau+h}f(\tau,y(\tau))d\tau \approx y(\tau)+hf\left(\tau+h\alpha_2,y+h\beta^1_2y\right) = y(\tau)+hf\left[\tau+\frac{1}{2}h,y(\tau)+\frac{h}{2}f(\tau,y(\tau))\right]
\end{gather*}
\end{document}