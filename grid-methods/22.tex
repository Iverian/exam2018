\documentclass[__main__.tex]{subfiles}

\begin{document}

\section{Методы Рунге-Кутта численного решения задачи Коши для нормальной системы ОДУ и для ОДУ, разрешённого относительно старшей производной}

Пусть $\;^>f = [f_1, f_2, ...,  f_n\rangle \in C^q([0; t] \times \;^>\mathbb{R}^n, \;^>\mathbb{R}^n)$ — гладкая векторная функция, где $f_i \in C^q([0; t] \times \;^>\mathbb{R}^n, \;^>\mathbb{R}^n)$ для $i = \overline{1, n}$.\\
Рассмотрим задачу Коши дл\ нормальной системы ОДУ:
\begin{equation}
	\begin{cases}
	\frac{d\;^>y}{d\tau} = \;^>f(\tau, \;^>y), \tau \in [a; b];
	\;^>y(a) = \;^>C_0 =
	\begin{pmatrix}
	c_1\\
	c_2
	.\\
	.\\
	.\\
	c_n
	\end{pmatrix}
	\in \;^>\mathbb{R}.
	\end{cases}
	\label{22-1}
\end{equation}
Для численного решения задачи $\ref{22-1}$ введём на $[a; b]$ схему равномерных сеток $A_{(.)} = (A_k)_{\mathbb{N}}$,\\
где $A_k = \langle a = \tau_0, \tau_1, ..., \tau_k = b\rangle$ и $h = stp(A_k) = \frac{b-a}{k}$. Согласно этой схеме сеток для численного решения задачи $\ref{22-1}$ используем метод Рунге-Кутта различных порядков $m \in \mathbb{N}$. Эти методы аналогичны методам Рунге-Кутта, которые используются для задач Коши с нормальным ОДУ.\\
\textbf{m = 1} Метод Эйлера\\
Конечно-разностная схема метода:
\begin{equation}
	\begin{cases}
	\;^>u_0 = \;^>c_0;\\
	\;^>u_{i+1} = \;^>u_i + h \;^>w_1(\tau_0 \;^>u_i; h).
	\end{cases}
	\label{22-2}
\end{equation}
\textbf{m = 2} Метод Эйлера-Коши
\begin{equation}
	y_{i+1} = y_i + \int \limits_{\tau_i}^{\tau_i + h} f(\tau, y(\tau))d\tau = y_i + h(\frac{1}{2}f(\tau_i, y_i) + \frac{1}{2}(\tau_i + h_i, y_{i+1})) =\\
	= y_i + h[\frac{1}{2}f(\tau_i, y_i) + \frac{1}{2}f(\tau_i+h, y_i+hf(\tau_i, y_i))] + o(h^2)
\end{equation}
Конечно-разностная схема:
\begin{equation}
	\begin{cases}
	\;^>u_0 = \;^>c_0;\\
	\;^>u_{i+1} = \;^>u_i + h(\frac{1}{2}\;^>w_1 + \frac{1}{2}\;^>w_2), \; i = \overline{0, k-1}.
	\end{cases}
\end{equation}
где 
\begin{equation}
	\begin{cases}
	\;^>w_1 = \;^w_1(\tau, \;^>u, h) = \;^>f(\tau, \;^>u) =
	\begin{pmatrix}
	f_1(\tau, \;^>u)\\
	.\\
	.\\
	.\\
	f_n(\tau, \;^>u)
	\end{pmatrix};\\
	\;^>w_2 = \;^>f(\tau+h, \;^>u + h\;^>w_1).
	\end{cases}
\end{equation}
\textbf{m = 4} Конечно-разностная схема:\\
\begin{equation}
	\;^>u_0 = \;^>c_0;\\
	\;^>u_{i+1} = \;^>u_i + h(\frac{1}{6}\;^>w_1  + \frac{2}{6}\;^>w_2 + \frac{2}{6}\;^>w_3 + \frac{1}{6}\;^>w_4), \; i = \overline{0, k-1},
\end{equation}
где для $\;^>u = \;^>u(\tau):$
\begin{equation}
	\begin{cases}
	\;^>w_1 = \;^>w_1(\tau, \;^>u, h) = \;^>f(\tau, \;^>u);\\
	\;^>w_2 = f(\tau + \frac{1}{2}h, \;^>u + \frac{1}{2}h\;^>w_1) = \;^>w_2(\tau, \;^>u, h);\\
	\;^>w_3 = f(\tau + \frac{1}{2}h, \;^>u + \frac{1}{2}h\;^>w_2) = \;^>w_3(\tau, \;^>u, h);\\
	\;^>w_4 = f(\tau + h, \;^>u + \;^>w_3) 
	\end{cases}
\end{equation}
\end{document}
