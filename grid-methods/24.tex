\documentclass[__main__.tex]{subfiles}

\begin{document}

\section{Неявные методы Эйлера и Эйлера-Коши для численного решения задачи Коши с нормальным ОДУ}
Рассмотрим задачу коши:
\begin{align*} 
\begin{cases}
\frac{dy}{dt} = f(t,y(t)) , \ t \in (0;T) \\
y(0) = y_0;
\end{cases}
\end{align*}
где $f$ достаточно гладкая функция
\subsection*{Неявный метод Эйлера}
Сетка $ A = <0=\tau_0,\tau_1,...,\tau_k>$ шага $ h = \frac{T}{k}$ \\
Если $y_n$ - приближенное решение задачи коши в узлу $\tau_n \in A$, где $0 \leq n \leq k$, то в неявном методе Эйлера \\
для получения конечно-разностной схемы используется соотношение: \\
$$ y_{n+1} = y_n + \int_{\tau_n}^{\tau_n+1} f(\tau, y(\tau)d\tau \approx y_n + hf(\tau_{n+1},y_{n+1})$$
Таким образом, в неявном методе Эйлера необходимо решать уравнение: \\
$$y = y_n + hf(\tau_{n+1},y) = \phi(y) $$
Для решения этого уравнения, поскольку $ \phi(y) = h\frac{df(\tau_{n+1},y)}{dy}$, можно использовать метод простой итерации, т.к. при достаточно малой $h$
выполняется условие: $|\phi'(y)| \leq 1 $. В качестве начального значения $y_{n+1}^{(0)}$ можно выбрать число $y_{n+1}^{(0)} = y_n + hf(\tau_n,y_n)$ из явного метода Эйлера.
Получив приближенное значение $y_{n+1}^{(p)}$ на p-ом шаге метода простых итераций для неявного метода Эйлера используют соотношение:
$$y_{n+1} = y_n + h(\tau_{n+1},y_{n+1}^{(p)}), n = 0,..k-1$$
где $y_n$ - приближенное значение решения задачи в узле $\tau_n \in A$
\subsection*{Неявный метод Эйлера-Коши}
В этом методе используется формула:
\begin{align*}
\begin{cases}
y_{n+1} = y_n + \int_{\tau_n}^{\tau_{n+1}}f(\tau,y(\tau))d\tau \approx y_n + \frac{h}{2}(f(\tau_n,y_n)-f(\tau_{n+1},y_{n+1})) \\
n = 0,k-1
\end{cases}
\end{align*}
Для приближенного вычисления $y_{n+1}$ используют метод простой итерации:
$$y = (y_n + \frac{h}{2}f(\tau_n,y_n)) + f(\tau_{n+1},y_{n+1})\frac{h}{2} = \psi(y) $$
Начальное условие: $y_{n+1}^{(0)} = y_n + hf(\tau_n,y_n)$ и $|\psi'(y)| \leq 1$ для достаточно малых h

\end{document}
