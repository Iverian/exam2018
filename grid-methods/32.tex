\documentclass[__main__.tex]{subfiles}

\begin{document}

\section{Спектральный признак устойчивости в явной разностной схеме для двумерного параболического уравнения}

Отсюда и до * интро. Восславим Рглора, чтобы оно было нужным.\\

Для дискретных функций, определенных на сетке
$$
m=mh;\ \ tn=n\tau
$$
$$u^n_m=e^{\omega\tau n+i k h m}=\lambda^n e^{i\alpha m}$$
Где $\lambda\in[0;2\pi]$
$$u^{n+1}_m=\lambda^{n+1} e^{i\alpha m}\Rightarrow u^{n+1}_m=\lambda u^n_m\Rightarrow u^n_m=\lambda^n u^0_m$$
А значит
$$max_m|u^n_m|=|\lambda|^n max_m|u^0_m|$$
Возпользовавшись определением устойчивости получим:
$$||u^{(h)}||_m\le C||f^{(h)}||_{f_n},\ \ c=const>0$$
Нормы: $||u^{(h)}||_{u_i}=max_{m,n}|u^n_m|,\ \ ||f^{(h)}||_{f_n}=max|u^0_m|$
Для устойчивости схемы необходимо, чтобы $f_n$ и для всех $|\lambda|^m\le C(n-\frac{1}{\tau})$
$$|\lambda(\alpha)| \le 1+C_1 \tau$$
*\\
Двумерное уравнение теплопроводности (параболическое)
$$\frac{\partial u}{\partial t}=K\left(\frac{\partial^2 u}{\partial x^2}+\frac{\partial^2 u}{\partial y^2}\right)$$
Где $K=const$\\
Мы будем искать решение в виде:
$$u^p_{m,n}=\lambda^p e^{i\alpha m} e^{i\beta n}$$
Получим:$$\frac{\lambda-1}{\tau}=K(-\frac{4}{n^2}\sin^2{\frac{\alpha}{2}}-\frac{4}{n^2}\sin^2{\frac{\beta}{2}})$$
$$\lambda=1-\frac{4K\tau}{n^2}(\sin^2{\frac{\alpha}{2}}+\sin^2{\frac{\beta}{2}})$$
$$|\lambda|<1\Rightarrow \frac{K\tau}{n^2}<\frac{1}{4}$$
А значит схема является устойчивой.
\end{document}
