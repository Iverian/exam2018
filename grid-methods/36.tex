\documentclass[__main__.tex]{subfiles}

\begin{document}

\section{Спектральный признак устойчивости в явных разностных схемах (правой и левой) для одномерного уравнения переноса}

Коротко и о главном.\\
Отсюда и до * интро. Восславим Рглора, чтобы оно было нужным.\\

Для дискретных функций, определенных на сетке
$$
m=mh;\ \ tn=n\tau
$$
$$u^n_m=e^{\omega\tau n+i k h m}=\lambda^n e^{i\alpha m}$$
Где $\lambda\in[0;2\pi]$
$$u^{n+1}_m=\lambda^{n+1} e^{i\alpha m}\Rightarrow u^{n+1}_m=\lambda u^n_m\Rightarrow u^n_m=\lambda^n u^0_m$$
А значит
$$max_m|u^n_m|=|\lambda|^n max_m|u^0_m|$$
Возпользовавшись определением устойчивости получим:
$$||u^{(h)}||_m\le C||f^{(h)}||_{f_n},\ \ c=const>0$$
Нормы: $||u^{(h)}||_{u_i}=max_{m,n}|u^n_m|,\ \ ||f^{(h)}||_{f_n}=max|u^0_m|$
Для устойчивости схемы необходимо, чтобы $f_n$ и для всех $|\lambda|^m\le C(n-\frac{1}{\tau})$
$$|\lambda(\alpha)| \le 1+C_1 \tau$$
*\\
Уравнение переноса: 
$$\frac{\partial u(t,x)}{\partial t}+C\frac{\partial u(t,x)}{\partial x}=f$$
Рассмотрим две схемы разложения:\\

Левая:$$\frac{u^{n+1}_m}-u^n_m{\tau}+C\frac{u^n_m-u^n_{m-1}}{h}=f^n_m$$
Правая:$$\frac{u^{n+1}_m}-u^n_m{\tau}+C\frac{u^n_{m+1}-u^n_{m}}{h}=f^n_m$$
Спектральный признак устойчивости:
$$u^n_m=\lambda^n e^{im\omega}$$
Подставим в правую схему:
$$\frac{\lambda-1}{\tau}+C\frac{e^{i\omega h}-1}{h}=0\Rightarrow\lambda=1+\frac{C\tau}{h}-C\tau \frac{e^{ih\omega}}{2h}$$
Так как $\omega$ - произвольна
$$|\lambda|\le 1 \ \ \forall \omega \Leftrightarrow -1\le \frac{C\tau}{h}\le 0$$
А значит схема устойчива при $C<0;\ \ \ \frac{|C\tau|}{h}\le 1$\\

Аналогично подставим в левую схему и получим:
$$\lambda=1+\frac{C\tau}{n}(1-e^{-i\omega h}$$
А значит схема устойчива при $C<0;\ \ \ \frac{|C\tau|}{h}\le 1$\\

\end{document}
