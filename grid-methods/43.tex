\documentclass[__main__.tex]{subfiles}

\begin{document}

\section{Единственность решения СЛАУ с трёхдиагональной матрицей, удовлетворяющей условию квази-диагонального преобладания}

\begin{definition}[условие квази-диагонального преобладания]
Если трехдиагональная матрица
$$
G_{(k)} = 
\left(
\begin{array}{cccccc}
B_0 & C_0 & 0 & 0 & \cdots & 0 \\
A_1 & B_1 & C_1 & 0 & \cdots & 0 \\
0 & A_2 & B_2 & C_2 & \cdots & 0 \\
\vdots & \vdots & \ddots & \ddots & \ddots & \vdots \\
0 & 0 & 0 & A_{k-1} & B_{k-1} & C_{k-1} \\
0 & 0 & 0 & 0 & A_{k} & B_{k} \\
\end{array}
\right)\in\mathrm{L}(\mathbb{R},k+1)
$$
удовлетворяет условиям $\forall j\in\overline{0,k}\colon |B_j|\ge|C_j|\wedge|B_j|\ge|A_j|+|C_j|>|A_j|$, то говорят, что $G_{(k)}$ имеет квази-диагональное преобладание.
\end{definition}

\begin{theorem}
Трехдиагональная матрица $G_{(k)}$ с квази-диагональным преобладанием невырождена.
\label{nonvirozd}
\end{theorem}
\begin{proof}
Рассмотрим СЛАУ $G_{(k)}\cdot\symbf{u}_{(k)} = \symbf{0}$, для ее решения используем метод прогонки:
$$
\left\{
\begin{gathered}
u_{j} = L_{j}u_{j+1}+M_{j}\hfill\\
u_{k} = M_{k}\hfill\\
\end{gathered}
\right.
\Longleftrightarrow
\left\{
\begin{gathered}
L_{0} = - \frac{C_0}{B_0}, \quad M_{0}=0
\hfill\\
L_{j} = \frac{C_j}{L_{j-1}A_j + B_j}
, \quad
M_j = -\frac{M_{j-1}A_j}{L_{j-1}A_j + B_j}
, \quad
\forall j\in\overline{1,k}
\hfill\\
M_{k}=-\frac{M_{k-1}A_k}{L_{k-1}A_k + B_k}
\hfill\\
\end{gathered}
\right.,
$$
покажем, что $\forall i\in\overline{0,k-1}\colon|L_i|\le 1$. Поскольку $|B_0|\ge|C_0|$, то $|L_0|=\frac{|C_0|}{|B_0|}\le 1$. Теперь положим $|L_{j-1}|\le 1$, покажем что $|L_j|\le 1$, где
$|L_j| = \frac{|C_j|}{|L_{j-1}A_j + B_j|}$ и $|L_{j-1}A_j + B_j| > |B_j| - |L_{j-1}||A_j| \ge \Big\lvzigzag |L_{j-1}| \le 1 \Big\rvzigzag \ge |B_j| - |A_j|\ge |C_j|$ в силу квази-диагонального преобладания $G_{(k)}$.

Следовательно $|L_{j}| = \frac{|C_j|}{|L_{j-1}A_j + B_j|}\le 1$ и $|L_{j_1}A_j+B_j|\ge |C_j|>0 \lvzigzag |A_j|+|C_j|>|A_j| \rvzigzag$.

Таким образом получим, что матрица $G_{(k)}$ невырождена.
\end{proof}

\begin{statement}
СЛАУ $G_{(k)}\cdot\symbf{u}_{(k)} = \symbf{0}$, где $G_{(k)}$ -- трехдиагональная матрица с квази-диагональным преобладанием имеет единственное решение.
\end{statement}
\begin{proof}
Прямое следствие из Теор. \ref{nonvirozd}.
\end{proof}

\end{document}
