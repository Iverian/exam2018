\documentclass[__main__.tex]{subfiles}

\begin{document}

\section{Принцип максимума для СЛАУ с трёхдиагональной матрицей}
Рассмотри СЛАУ:
\begin{align*}
\begin{cases}
U_0 = U_a \\
a_iU_{i-1} + b_iU_i + c_iU_{i+1} = d_i , \ \ \ 1 \leq i \leq N \\
U_n = U_b
\end{cases}
\end{align*}
где $a_i \leq 0$, $c_i < 0$, $b_i > 0$, а также $a_i+b_i+c_i \geq 0$. \\
Из этого условия следует справедливость условия квази-диагонального преобладания для матрицы СЛАУ: \\
$|b_i| \geq |a_i|+|c_i| $, $|b_i|\geq|a_i|$, $i \leq 1 < N$.\\
Следовательно, СЛАУ имеет единственное решение
\begin{theorem}(Принцип максимума для СЛАУ с трёхдиагональной матрицей)\\
Пусть СЛАУ удовлетворяет этим условиям и, кроме того, $U_a \leq 0$, $U_b \leq 0$ и $d_i \leq 0$ для $ i \leq N$.\\
Тогда решение $U = [U_0,U_1,...,U_N>$ СЛАУ удовлетворяет условиям $U_i \leq 0$ для $ i \leq N$ \\
Доказательство (от противного): \\
Предположим обратное, тогда для индексов $i \in 1,N-1$ выберем максимальный индекс $j$ для которого достигается максимум значения $U_j$ 
среди чисел $U_1,U_2,...,U_{j-1},..,U_{N-j}$\\
Тогда, из условий:
$a_i \leq 0$, $c_i < 0$, $b_i > 0$, а также $a_i+b_i+c_i \geq 0$ \\
Получаем:
$ a_jU_{j-1} \geq a_j U_j ; c_jU_{j+1} > c_jU_j$\\
и
$$ 0 \leq a_jU_j+b_jU_j+c_jU_j = (a_j+b_j+c_j)U_j < a_jU_{j-1} + b_j U_j + c_j U_{j+1} = d_j \leq 0$$
Получаем противоречие.
\end{theorem}


\end{document}
