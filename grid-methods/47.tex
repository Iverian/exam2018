\documentclass[__main__.tex]{subfiles}

\begin{document}

\section{Вид конечно-разностной схемы при численном решении краевой задачи для одномерного стационарного уравнения теплопроводности с переменным коэффициентом теплопроводности}

Рассмотрим стационарное уравнение теплопроводности:

\begin{gather}
	\begin{cases}
		-\frac{d}{d\tau}(k(\tau)\frac{du}{d\tau})+v(\tau)\frac{du}{d\tau}=0, \tau \in (a;b); \\
		u(0)=u_{a}, u(b)=u_{b}.
	\end{cases}
\end{gather}

где $k(\tau)>k_{0}>0 \  q(\tau) \geq 0.$

Конечно-разностная схема для данной задачи:
\begin{gather}
	\begin{cases}
		^>u_{\cdot}=[u_{\cdot}, u_{1},...,u_{n}>\in ^>\mathbf{R}^{|A|}(A)\\
		u_{0}=u_{a},\\
		-\frac{1}{h}(k_{i+\frac{1}{2}}\frac{u_{i+1}-u_{i}}{h}-k_{i-\frac{1}{2}}\frac{u_{i}-u_{i-1}}{h})+v_{i}\frac{u_{i+1}-u_{i-1}}{2h}=0, \ i=1,...,k-1,\\
		u_{k}=u_{b}.
	\end{cases}
\end{gather}
где $\tau_{i-\frac{1}{2}}=\frac{\tau_{i-1}+\tau_{i-1}}{2}, k_{i-\frac{1}{2}}=k(\tau_{i-\frac{1}{2}}),\tau_{i+\frac{1}{2}}=\frac{\tau_{i+1}+\tau_{i}}{2}, k_{i+\frac{1}{2}}=k(\tau_{i+\frac{1}{2}})$ для i=1..k-1. 



\end{document}
