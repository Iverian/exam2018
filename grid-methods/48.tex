\documentclass[__main__.tex]{subfiles}

\begin{document}

\section{Аппроксимация со вторым порядком краевых условий 2-го рода разностной схемы краевой задачи для одномерного стационарного уравнения теплопроводности}

Рассмотрим краевую задачу:

\begin{equation}
\label{eq_48_1}
\begin{cases}
-frac{d}{dt}(k(\tau) \frac{du}{dt}) + q(\tau)u(\tau) = f(\tau), \tau \in (a; b);\\
-k(a)u'(a) = w_a; u(b) = u_b;
\end{cases}
\end{equation}

В задаче \ref{eq_48_1} левое краевое условие является условием 2-го рода, правая - 1-го рода.
Приведем метод аппроксимирования левого краевого условия задачи \ref{eq_48_1} со 2-ым порядком аппроксимирования. Для правой разностной производной имеем:

\begin{equation}
\label{eq_48_2}
\frac{u(a + h) - u(a)}{h} = u'(a) + u''(a)\frac{h}{2} + u'''(a + \Theta h)\frac{h^2}{6}\;\;\;\;\{0 < \Theta < 1\}
\end{equation}

Уравнение из задачи \ref{eq_48_1} представляем в виде:

$$-k(\tau)u'' - k'(\tau)u'(\tau) + q(\tau)u(\tau) = f(\tau)$$

Отсюда при $\tau = a$ получаем:

\begin{equation}
\label{eq_48_3}
-k(a)u''(a) = k'(a)u'(a) - q(a)u(a) + f(a)
\end{equation}

Используя \ref{eq_48_2} и \ref{eq_48_3} получаем:

$$-k(a)\frac{u(a + h) - u(a)}{h} = -k(a)u'(a) + \frac{h}{2}\left[\frac{k'(a)}{k(a)} \cdot k(a) u'(a) - q(a) u(a) + f(a) \right] + O(h^2) \iff$$
$$\iff -k_0 \frac{u_1 - u_0}{h} = w_a + \frac{h}{2}\left[-\frac{k'(a)}{k_0} - q_0 u_0 + f_0\right] + O(h^2), h \to 0$$

Таким образом, левое краевое условие \ref{eq_48_1} аппроксимируется со 2-ым порядком аппроксимирования в виде:

$$k_0 \frac{u_1 - u_0}{h} + \frac{q_0 h}{2} u_0 = w_a + \frac{h}{2} \left[ - \frac{k'(a)}{k_0} w_a + f_0 \right]$$
 
\end{document}
