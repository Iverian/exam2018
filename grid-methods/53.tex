\documentclass[__main__.tex]{subfiles}

\begin{document}

\section{Итерационный метод для численного решения конечно-разностного аналога в разностной схеме для задачи Дирихле с прямоугольной областью}

Рассмотрим двумерную задачу Дирихле:
\begin{gather*}
\left\{
\begin{gathered}
\frac{\partial^2 u}{\partial x^2} + \frac{\partial u^2}{\partial y^2} = f(x,y)
\quad (x,y)\in\symbf{\Omega}
\hfill\\
\left.u(x,y)\right|_{\symbf{\partial\Omega}} = g(x,y)
\hfill\\
\end{gathered}
\right.,
\end{gather*}
где $\symbf{\Omega}=(a,b)\times(c,d)$ -- прямоугольная область, а $\symbf{\partial\Omega}$ -- е граница. Введем на $\symbf{\Omega}$ сетку $C$ размера $n \times m$:

$$
C = A \times B = \langle \left.(x_i,y_j)\right|\forall i\in\overline{0,n-1},j\in\overline{0,m-1}\colon x_i = a + ih_x \wedge y_j = b + jh_y\rangle,
$$

где $h_x=\frac{b-a}{N-1}$ и $h_y=\frac{d-c}{M-1}$. Перейдем к конечно-разностному аналогу:

\begin{gather*}
\left\{
\begin{gathered}
\frac{u_{i-1 \ j} -2u_{ij}+u_{i+1 \ j}}{h_x^2} + \frac{u_{i \ j-1} - 2u_{ij} + u_{i \ j+1}}{h_y^2}
=
f_{ij}
\quad
i\in\overline{0,n-1},j\in\overline{0,m-1}
\hfill\\
u_{i\ 0} = g(x_i, c)
\quad
i\in\overline{0,n-1}
\hfill\\
u_{0\ j} = g(a, y_j)
\quad
j\in\overline{0,m-1}
\hfill\\
u_{i\ m-1} = g(x_i, d)
\quad
i\in\overline{0,n-1}
\hfill\\
u_{n-1\ j} = g(b, y_j)
\quad
j\in\overline{0,m-1}
\hfill\\
\end{gathered}
\right.
\end{gather*}

Элемент $u_{ij}$ выражается из конечно-разностного уравнения:

$$
u_{ij}
=
\frac
{h_y^2(u_{i-1\ j} + u_{i+1\ j}) + h_x^2(u_{i\ j-1} + u_{i\ j+1}) - h_x^2h_y^2f_{ij}}
{2(h_x^2+h_y^2)}
$$

Итеративные методы решения такой схемы строятся на выборе начального приближения $\forall i\in\overline{1,n-2}\forall j\in\overline{1,m-2}\colon u\indices{^0_i_j}$ для внутренних точек сетки и последовательного вычисления приближений $u\indices{^k_i_j}$ по определенному правилу пока не выполняется условие останова $\left\Vert U^{(k+1)} - U^{(k)}\right\Vert < \varepsilon$.

Например, \textit{метод Гаусса -- Зейделя} предлагает следующее правило:

$$
u\indices{^k_i_j}
=
\frac{
h_y^2(u\indices{^k_{i-1\ }_j} + u\indices{^{k-1}_{i+1\ }_j})
+
h_x^2(u\indices{^k_{i\ }_{j-1}} + u\indices{^{k-1}_{i\ }_{j+1}})
-
h_x^2h_y^2f_{ij}}{2(h_x^2+h_y^2)}
$$

Можно показать, что при $h=h_x=h_y$ этот метод равномерно сходится к решению, а погрешность имеет порядок $h^2$.

\end{document}
