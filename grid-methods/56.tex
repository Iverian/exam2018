\documentclass[__main__.tex]{subfiles}

\begin{document}

\section{Метод Ритца численного решения одномерной стационарной задачи для уравнения теплопроводности}

231 страница лекций Димы.\\

На $[a,b]$ заданы непрерывные функции $k>0,q\ge 0,\ \ f$. Кроме того на $[a,b]$ рассматривается множество кусочно гладких функций $\mathbb{U}$, удовлетворяющих следующим условиям:\\
Если $u\in \mathbb{U}$, то $u(a)=u_a,\ u(b)=u_b$\\
На множестве $\mathbb{U}$ определен функционал $\mathbb{J}:\mathbb{U}\rightarrow\mathbb{R}$ вида $\int_a^bF(x,u,u')dx$\\
Где $F(x,u,u')=\frac{1}{2}(k(u')^2+qu^2)-fu$\\
Ставится задача 
\begin{equation}
\begin{cases}
\mathbb{J}(u)\rightarrow min\\
u\in \mathbb{U}\\
\end{cases}
\end{equation}
Решение этой задачи должно удовлетворять уравнению Эйлера:
$$-\frac{d}{dx}F_{u'}+F_u=0$$
Из чего получаем:
\begin{equation}
\begin{cases}
-(ku')'+qu-f=0\\
u(a)=u_a,\ u(b)=u_b\\
\end{cases}
\end{equation}
Таким образом для решения краевой задачи можно решать задачу, поставленную в первой системе, где функционал $\mathbb{J}$ положительно определен.\\

Для численного решения на $[a,b]$ этой задачи вводится сетка $A=<a=x_0,x_1,..,x_N>$ шага $stp(A)=h=\frac{b-a}{N}$. Кроме того выбирая стандартный базис сплайнов 1-ой степени пространства $Spl_1(A)$ вида:
$$H=<h_0,h_1,..,h_N>\text{ где } h_i=spl_1(A;e_{i+1}) \text{ для } i = \hat{0,N}$$
Приближения решения задачи представляется в виде:
$$ u =u_a h_0+\sum_{i=1}^{N-1}(u_i h_i)+u_b h_N $$
, где коэффициенты $u_i$ неизвестны.\\
Таким образом, фактически, для множества функций $\mathbb{U}$ выбирается аппроксимация $\hat{p}:\mathbb{U}\rightarrow^>\mathbb{R}^{|A|}(A)$, для которой, елси $\hat{A}(u)=^>u$, то $\hat{p}=u_a h_0+\sum_{i=1}^{N-1}(u_i h_i)+u_b h_N$.\\

Аппроксимирование множества $\mathbb{U}; N\rightarrow\infty$, построенное на таких аппроксимациях корректно.\\

$$\mathbb{J}(u)=\frac{1}{2}\int_a^b(k(u')^2+qu^2)dx-\int_a^b fu dx$$ 
Или 
$$\mathbb{J}(u)=\frac{1}{2}\int_a^b(k\sum_{i=0}^{N}\sum_{j=0}^{N}(u_j h_j')(u_i h_i')+q\sum_{i=0}^{N}\sum_{j=0}^{N}(u_j h_j)(u_i h_i))dx-\int_a^b f\sum_{i=1}^{N-1}(u_i h_i) dx$$
Сворачивая суммы богоугодным обазом и сжигая пленников во славу Рглора получим:
$$\frac{1}{2}^<v A\ ^>v-^<v^>g+const$$
Где:
$$^>v=[u_1,...,u_{N-1}>$$
$$A=(a_{ij})_{(N-1)\times(N-1)}$$
$$a_ij=\int_a^b(kh_i'h_j'+qh_i h_j)dx$$
$$g_i=\int_a^b(f h_i-(u_a+u_b)\sum_{j=1}^{N-1}(kh_j'+qh_i))dx$$
Новую задачу 
\begin{equation}
\begin{cases}
\mathbb{J}(^>v)\rightarrow min\\
^>v\in ^>\mathbb{R}^{N-1}\\
\end{cases}
\end{equation}
решаем через поиск минимума конечномерного линейноквадратичного функционала (230 страница лекций димы и хз какой билет, но вроде не этот)\\

Поскольку $h_ih_j=0,$ если $|i-j|>0$, то матрица $A$ 3-х диагональна и положительно ориентирована.
\end{document}
